\chapter{Mechanik}

\renewcommand{\kapitelautor}{Autor: Alexander Punz}

\section{Allgemeine technische Planung}
	
		\subsubsection{Allgemeine Informationen über 3D Drucken}

Die Technologie des 3D Druckens hat in den letzten Jahren immer mehr an Popularität gewonnen. Mit Hilfe des 3D Druckers kann man fast alle vorstellbaren Formen anfertigen. Es gibt verschiedenste Verfahren wie man ein Werkstück anfertigen kann: Laser Sintern, Stereolithographie, Drucken mit flüssigen Materialien, etc. In diesem Projekt wird nur die Variante des Druckens mit flüssigen Material verwendet. Diese ist kostengünstig bzw. genau genug für die Teile. Wie der Name schon sagt, wird Material in einem Druckkopf geschmolzen und dann Schicht für Schicht auf der Druckplatte aufgetragen. Der Druckkopf fährt nur in X und Y Richtung, die Höhe wird von der Druckplatte selbst bestimmt.
Je nachdem wie der Drucker aufgebaut ist, werden die Produkte genau oder nur grob angefertigt. Sehr genaue Teile kann man am besten in einem Drucker produzieren, der einen geschlossenen Druckraum bzw. eine beheizte Druckplatte hat. Besonders an dünnen Platten merkt man das. Wenn der Druckraum nach bzw. während des Druckvorgangs warm ist, kühlt das Werkstück an jeder Stelle fast gleich ab. Ist der Druckraum offen, kühlt das Werkstück in der Mitte schneller ab, kühles Material zieht sich zusammen, daher biegt sich das Material auf.
Es kann vorkommen, dass ein Teil nur so gedruckt werden kann, wenn es nicht komplett auf der Druckplatte aufliegt zum Beispiel ein Steg, der „in der Luft“ oder eine Bohrung im Werkstück, die horizontal gedruckt werden muss. In solchen Fällen, druckt der Drucker unter diesem Steg Stützmaterial. Basierend auf diesem Stützmaterial, wird dann die gewünschte Form gedruckt. Das Stützmaterial ist so gefertigt, dass man es leicht von dieser abbrechen kann, ohne dass Rückstände zurück bleiben.

		\subsubsection{Materialeigenschaften}

		\subsubsection{Von der Idee zur Anfertigung}

\section{Halterung für Cupcakes}

	\subsection{Technische Planung}

	\subsection{Umsetzung}

	\subsection{Herausforderungen und Lösungen}

	\subsection{Implementierung}

\section{Rotorschutz}

	\subsection{Technische Planung}

	\subsection{Umsetzung}

	\subsection{Herausforderungen und Lösungen}

	\subsection{Implementierung}

\section{Halterung Ultraschallsensor}

	\subsection{Technische Planung}

	\subsection{Umsetzung}

	\subsection{Implementierung}

\section{Halterung PIXY cmucam5}

	\subsection{Technische Planung}

		\subsubsection{Berechnungen}

	\subsection{Umsetzung}

	\subsection{Implementierung}

	\subsection{Testphase}

\section{Führung für Testflüge}

	\subsection{Technische Planung}

	\subsection{Umsetzung}

	\subsection{Implementierung}

	\subsection{Testphase}

\section{Persönliche Erfahrungen}