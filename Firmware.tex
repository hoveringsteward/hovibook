\chapter{Firmware}
\renewcommand{\kapitelautor}{Autor: Lucas Ullrich}

%%%%%%%%%%%%%%%%%%%%%%%%%%%%%%%%%%%%%%%%%%%%%%%%%%%%%%%%%%%%%%%%%%%%%%%%%%%%%%%
\section{Allgemeine technische Planung}

  \subsection{Tischkonzept}

  \subsection{Flussdiagramme}

  \subsection{Tools}

    \subsubsection{GitHub}

    \subsubsection{MPLAB}

%%%%%%%%%%%%%%%%%%%%%%%%%%%%%%%%%%%%%%%%%%%%%%%%%%%%%%%%%%%%%%%%%%%%%%%%%%%%%%%
\section{Navigation}

  \subsection{Technische Planung}

  \subsection{Umsetzung}

    \subsubsection{Vergleichen der Frames}
    Für den Vergleich des aktuellen mit dem letzten Frame, werden zwei Strukturen verwendet, die über folgende Mitglieder verfügt.
    \begin{itemize}
      \item \textbf{num}\\
      Num ist die ID des getrackten Colorcodes, er besteht aus einer zweistelligen Zahl.
      \item \textbf{pos\_x}\\
      Pos\_x ist die X-Position des Colorcodes. Der Wert bezieht sich auf das Zentrum des Objektes.
      \item \textbf{pos\_y}\\
      Pos\_y ist die Y-Position des Colorcodes. Der Wert bezieht sich auf das Zentrum des Objektes.
      \item \textbf{height}\\
      Height ist die, vom Ultraschall übergebene Höhe.
      \item \textbf{angle}\\
      Angle ist die Rotation des Colorcodes. Da er zweifarbig ist, kann die Pixy CMUcam5 die Rotation des Objektes feststellen.
    \end{itemize}

    Zuerst wird die ID des Colorcodes verglichen, um herauszufinden, ob das Farbobjekt das selbe wie im letzten Frame ist.
    Sollte dies der Fall sein, werden die Koordinaten x und y und die Rotation mit den Werten der älteren Struktur verglichen und gespeichert.
    Diese werden bei den folgenden Funktionen verwendet, um zu überprüfen, ob der Hexacopter die richtige Geschwindigkeit hat.

    \subsubsection{Aileron, Elevator und Rudder anhand der Kameradaten}
    Durch die Pixy CMUcam5 kann die Position des Hexacopters, relativ zu einem Colorcode, festgestellt werden. Gegebenenfalls werden die Flugparameter verändert.

    Die Pixy CMUcam5 XXXXXX % IRGENDWAS MIT X diese länge, y diese länge; %



      \begin{itemize}
        \item \textbf{Überprüfen von Aileron}\\
<<<<<<< HEAD
        Die Überprüfung von Aileron bezieht sich auf die Beschleunigung nach Links und Rechts, was der x-Koordinate entspricht.

        Ziel der Funktion ist es, den Farbcode in die Mitte des Frames zu bekommen. Der Idealzustand befidnet sich zwischen 150 und 170.
        Sollte dieser Zustand erreicht werden, bleibt der Wert von Aileron unverändert und der Hexacopter fliegt weiterhin mit dieser Beschleunigung an der
        x-Koordinate.
        Sollte dies nicht der Fall sein, muss der Wert auf einige Komponenten XXXXX überprüft werden.
        Wenn der Farbcode zu weit auf der rechten Seite liegt, das bedeutet, wenn der Wert des Mittelpunktes vom Farbobjekt höher als 170 ist,
        muss der Hexacopter nach rechts fliegen, um seine Position zu korrigieren.
        Dabei muss zuerst verglichen werden, ob sich der Hexacopter in die richtige Richtung bewegt. Sollte er in die falsche Richtung
        fliegen, wird der Aileron-Flugparameter gesenkt, was die Beschleunigung nach Rechts vorraussetzt.
        Sollte die Drohne bereits nach Rechts fliegen, wird je nach Geschwindigkeit





=======
>>>>>>> 035b7c4a5e4363783a37e9fc592292b20fc5f3c3
        \item \textbf{Überprüfen von Elevator}\\
        Position der Farbobjekte (CHECK AILERON \& ELEVATOR)
        Wenn ein Farbobjekt nicht im gewünschten Bereicht plaziert ist, muss der Hexacopter weiter nach links oder weiter nach rechts fliegen.
        \item \textbf{Überprüfen von Rudder}\\
        Rotation der Farbojekte (CHECK RUDDER)
        Wenn der Hexacopter über einem Farbobjekt fliegt, soll er kontrollieren, ob der 2-Farbige Code die richtige Rotation hat und sich im richtigen Bereich des Bildschirmes befindet. Wenn diese Informationen richtig sind, darf der Copter zum nächsten Farbobjekt fliegen.
        Durch dieses Sytem könne die genauen Wege vorgegeben werden und können sich durch das gesamte Restaurant verteilen. Durch die rotation der Codes können auch Kurven eingebaut werden.

        Richtung (CHECK RUDDER)
        Durch die Rotation der Colorcodes, kann der Hexacopter bestimmen, ob er den richtigen Weg und in die richtige Richtung fliegt, wenn er am Weg zurück zur Base ist, muss er den umgekehrten Colorcode verwenden. (Rotation 180Grad)
      \end{itemize}

    \subsubsection{Throttle anhand des Ultraschallsensors}

    Starten und landen auf Landeplattformen (CHECK THROTTLE)
    Der Hexacopter startet und landet auf den mit ebenfalls mit Colorcodes gekennzeichneten Landeplattformen.
    Bei einem Fehler, besteht auch das Landen auf einem beliebigen Fleck

    Höhe korrigieren (CHECK THROTTLE)
    Höhenunterschied zwischen Tisch und Boden

    \subsubsection{Speichern der alten Daten}


%%%%%%%%%%%%%%%%%%%%%%%%%%%%%%%%%%%%%%%%%%%%%%%%%%%%%%%%%%%%%%%%%%%%%%%%%%%%%%%
\section{Objekterkennung}

  \subsection{Technische Planung}

  \subsection{Umsetzung}

  \subsection{Herausforderungen und Lösungen}

%%%%%%%%%%%%%%%%%%%%%%%%%%%%%%%%%%%%%%%%%%%%%%%%%%%%%%%%%%%%%%%%%%%%%%%%%%%%%%%
\section{Sicherheit}

  \subsection{Technische Planung}

  \subsection{Umsetzung}

  \subsection{Herausforderungen und Lösungen}

%%%%%%%%%%%%%%%%%%%%%%%%%%%%%%%%%%%%%%%%%%%%%%%%%%%%%%%%%%%%%%%%%%%%%%%%%%%%%%%
\section{Systemausfall}

  \subsection{Technische Planung}

  \subsection{Umsetzung}

  \subsection{Herausforderungen und Lösungen}
