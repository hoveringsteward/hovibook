% !TEX root = diplomarbeit.tex
\chapter{Einleitung}
\renewcommand{\kapitelautor}{Autor: Markus Kaiser}

%%%%%%%%%%%%%%%%%%%%%%%%%%%%%%%%%%%%%%%%%%%%%%%%%%%%%%%%%%%%%%%%%%%%%%%%%%%%%%%
\section{Projektidee}

%%%%%%%%%%%%%%%%%%%%%%%%%%%%%%%%%%%%%%%%%%%%%%%%%%%%%%%%%%%%%%%%%%%%%%%%%%%%%%%
\section{Ausgangssituation}
  Im den folgenden Absätzen wird beschrieben, wie die Idee von Hovering Steward, dem autonom fliegenden Kellner
  entstanden ist, und womit sich die ersten Recherchen zu Begin des Projektes beschäftigt, beziehungsweise
  welche Ergebnisse diese herausgebracht haben.

  \subsection{Ideenfindung}
  Ihren Anfang fand die Idee im Projektmanagement Unterricht. Die Teams mussten ein fiktives Projekt erfinden, um auf Basis dieses
  Projektmanagementpläne zu erstellen. Es entstand "Fluorescent Bakery", eine Bäckerei, die fluoreszierende Cupcakes verkauft.
  Der technische Teil ergab sich später, bei dem Gedanken diese Idee für die Diplomarbeit weiterzuverwenden. Der Ansatz hierfür war,
  einen nicht menschlichen Kellner zu erschaffen, der mithilife künstlicher Intelligenz die Aufgaben einer Bedienung in einem Restaurant übernimmt.
  Einen Kellner auf Rädern erschien zu simpel, daher wurde die Entscheidung getroffen, die 3. Dimension mit einzubeziehen und ihn fliegen zu lassen.
  So entstand "Hovering Steward - der autonom fliegende Kellner", ein dreidimensionales Tracking-System, welches eine Drohne in einem Raum die Aufgaben eines Kellners durchführen lässt.
  Das Projekt erhielt später noch eine weitere Komponente, nämlich eine digitale Speisekarte, um das System in einem Restaurant vollständig zu automatisieren.

  \subsection{Was es schon gibt}

  \textbf{Themenrestaurants}
  \begin{itemize}
    \item Disaster Café\\
    Das Themenrestaurant Disaster Café in Spanien bietet Gästen die Erfahrung, ihr Essen bei einem Erdbeben
    der Stärke 7.8 zu sich zu nehmen.

    \item Das stille Örtchen - Modern Toilet\\
    Eingerichtet wie eine Toilette, können Gäste des Modern Toilet in Japan ihre Speisen in einem gewohnten Umfeld genießen.

    \item Affen-Kellner\\
    Die kleine Taverne in Japan besitzt zwei kleine Affen, die Tätigkeiten wie das Servieren von Getränken oder Handtüchern
    für den Inhaber übernehmen. Sie verstehen sogar, welche Getränke ein Gast bestellt.

    \item The Royal Dragon\\
    Neben Rollschuh fahrenden Kellnern verfügt das weltweit größte Restaurant für Meeresfrüchte eine
    Art Seilbahn, mithilfe welcher ein Kellner "fliegend" das Essen serviert.

    \item Dinner in the Sky\\
    Bei Dinner in the Sky werden dee Gäste samt Tisch und kleiner Küche von einem Kran 50 Meter
    in die Luft gezogen, wo dann gespeist wird.

    \item Dinner in the Dark\\
    In diesem Themenrestaurant verbringen die Gäste ihren Aufenthalt im Dunkeln. Einzig und allein die
    Kellner können mittels Nachtsichtbrille etwas sehen.
  \end{itemize}

  \textbf{Multicopter in der Gastronomie}
  \begin{itemize}
      \item{Infinium-Serve}\\
      Das Unternehmen Infinium-Robotics hat ein System entwickelt, welches Gästen eines Restaurants
      das Essen mittels Hexacopter serviert.

      \item{iTray}\\
      iTray, ein Projekt aus London, nutzt kleine Drohnen um Sushi an die Tische der Gäste zu bringen.
      Die Drohnen werden über ein Remote-Wi-Fi-System gesteuert.

  \end{itemize}

%%%%%%%%%%%%%%%%%%%%%%%%%%%%%%%%%%%%%%%%%%%%%%%%%%%%%%%%%%%%%%%%%%%%%%%%%%%%%%%
\section{Team und Aufgabenverteilung}
  \textbf{Markus Kaiser}\\
  \textbf{Projektleitung und Marketing}\\
  Markus Kaiser leitete das Team Hovering Steward und war somit für das Projektmanagement und
  die organisatorischen Aspekte verantwortlich. Neben dieser Hauptrolle zählte außerdem das Marketing
  zu seinen Aufgabenbereichen, was speziell die Entwicklung des Blogs anbelangte. Durch seine Kenntnisse
  in der Webentwicklung stellte er mit dem Blog eine wichtige Schnittstelle zur Außenwelt her.

  \textbf{Lucas Ullrich}\\
  \textbf{Sensorik \& Firmware}\\
  Lucas Ullrich war neben seiner Position als Projektleiter Stellvertreter für die Sensorik an unserer Drohne und
  für die Programmierung der Firmware zuständig. Er unterstützte unseren Projektleiter bei terminlichen Angelegenheiten
  und konnte durch seine Kenntnisse mit der Programmiersprache C eine solide Basis für die Firmware des Microcontrollers schaffen.
  Außerdem

  \textbf{Christina Bornberg}\\
  \textbf{Firmware}\\
  Christina Bornberg fungierte als Firmware Entwicklerin des Teams. Durch ihre Interesse an der Programmierung von Drohnen
  und bereits gesammelten Erfahrung mit der Programmierung von C konnte sie gemeinsam mit Lucas Ullrich die LOgik hinter
  dem automatisierten Flug der Drohne realisieren.

  \textbf{Katharina Joksch}\\
  \textbf{Webentwicklung}\\
  Der Aufgabenbereich von Katharina Joksch war die Webentwicklung, genauer gesagt die Programmierung der digitalen Speisekarte.
  Neben der Planung der Datenbank und der sinnvollen Verwendung hilfreicher Frameworks entwickelte sie außerdem eine Java Applikation,
  die die Kommunikation zwischen der Speisekarte und der Drohne regelte.

  \textbf{Alexander Punz}\\
  \textbf{Hardware \& Mechanik}\\
  Alexander Punz war sowohl für die Hardware als auch für die Mechanik verantwortlich. Seine Aufgaben waren sowohl die Konzeption
  und Produktion des Rotorschutzes, der einen sicheren Flug der Drohne ermöglichte, als auch die Herstellung diverser Halterungen,
  die für Sensorik, Transport und Flugtests verausgesetzt waren.

%%%%%%%%%%%%%%%%%%%%%%%%%%%%%%%%%%%%%%%%%%%%%%%%%%%%%%%%%%%%%%%%%%%%%%%%%%%%%%%
\section{Betreuer}
  \textbf{Mag. Andreas Fink}\\
  Mag. Andreas Fink stand dem Projektteam als Hauptbetreuer der Abteilung für Informationstechnologie zur Seite.
  Seine objektive Sichtweise auf das Projekt, hat dem Team sehr geholfen den Fokus auf die Ziele zu legen und
  das Projekt in die Richtung zu entwickeln.
  Zusätzlich dazu betreute er individuell Markus Kaiser bei den Aspekten Projektleitung und Marketing.

  \textbf{DI Herbert Fleck}\\
  DI Herbert Fleck war Hauptbetreuer der Mechatronik Abteilung unseres Teams. Er koordinierte den Prozess der Diplomarbeit
  gemeinsam mit Mag. Andreas Fink. Das Teeam schätzte außerdem sehr das Konstruktive Feedback bei Präsentationen.
  Er betreute nebenbei Lucas Ullrich mit Fachwissen aus dem Bereich der Elektronik.

  \textbf{DI August Hörandl}\\
  DI August Hörandl fungierte als Individualbetreuer von Christina Bornberg. Duch seine Fähigkeiten und
  Erfahungen als Programmierer sowohl mit der Sprache C, als auch Java war das Entwickeln der Firmware,
  aber auch der Java-Applikation für die WLAN-Kommunikation wesentlich einfacher. Außerdem war
  DI August Hörandl unser Ansprechpartner wenn es Unklarheiten bei LaTeX gab.

  \textbf{MMag. Florian Weiss}\\
  MMag. Florian Weiss betreute Katharina Joksch bei der Entwicklung der digitalen Speisekarte. Sein umfangreiches
  Know-How im Bereich der Webentwicklung, dem Umgang mit diversen Frameworks und Bibliotheken half Katharina
  dabei ein Grundgerüst für die Entwicklung aufzubauen.

  \textbf{DI Franz Temper}\\
  DI Franz Temper unterstützte Alexander Punz bei der Enticklung der Konstuktionen.

%%%%%%%%%%%%%%%%%%%%%%%%%%%%%%%%%%%%%%%%%%%%%%%%%%%%%%%%%%%%%%%%%%%%%%%%%%%%%%%
\section{Partner / Sponsoren}


%%%%%%%%%%%%%%%%%%%%%%%%%%%%%%%%%%%%%%%%%%%%%%%%%%%%%%%%%%%%%%%%%%%%%%%%%%%%%%%
\section{Danksagung}

\chapter{Projektmanagement}
\renewcommand{\kapitelautor}{Autor: Markus Kaiser}

%%%%%%%%%%%%%%%%%%%%%%%%%%%%%%%%%%%%%%%%%%%%%%%%%%%%%%%%%%%%%%%%%%%%%%%%%%%%%%%
\section{Ziele}

  \subsection{Muss-Ziele}
  \textbf{RE-M 01 Blog}\\
  Die Diplomarbeitswebsite fungiert in erster Linie als selbst programmierter Blog, um Interessenten
  jederzeit die Möglichkeit zu bieten, sich über den Status des Projektes zu informieren. Jedes Teammitglied
  kann individuelle Blog- oder Entwicklertagebucheinträge verfassen.

  \textbf{RE-M 02 Facebookauftritt}\\
  Eine Facebookseite names "Hovering Steward" ist erstellt. Sie informiert Interessenten über das
  Projekt und den Blog. Sponsoren sind genannt und das Diplomarbeitsteam ist vorgestellt.

  \textbf{RE-M 03 Sponsoren und Finanzen}\\
  Um die Kosten des Projektes zu decken, sind diverse Unternehmen, die möglicherweise Interesse an dem
  Projekt haben, per e-Mail oder telefonisch kontaktiert, und um eine Partnerschaft gefragt worden.
  Für kooperative Firmen sind diverse webetechnische Gegenmaßnahmen geplant worden, um das Sponsoring
  zu decken.

  \textbf{RE-M 04 Abrufbarkeit auf Tablets}\\
  Die digitale Speisekarte steht für Gäste auf Tablets zur Verfügung. Damit der Gast bestellen kann,
  ruft der User die Anwendung auf. Das Tablet benötigt zum Abruf der App mindestens einen der Browser
  Safari 7, Firefox 30 oder Internet Explorer 11.

  \textbf{RE-M 05 Speiseinformationen}\\
  Damit die Speisekarte der anerkannten Norm entspricht, enthält sie Informationen über die Inhaltsstoffe
  beziehungsweise Allergikerinformationen. Auf dem Screen der Speiseinformationen befindet sich ein
  Bestellbutton.

  \textbf{RE-M 06 Datenverwaltung}\\
  Die Daten der Speisekarte ruft die Webapp über eine Datenbank ab. Die Schnittstellen stehen für eine
  HTML und JSON Ausgabe zur Verfügung.

  \textbf{RE-M 07 Bestellfunktion}\\
  Durch den Klick auf den "Bestellbutton", der sich auf dem Speiseinformationsscreen befindet, wird eine
  Speise bestellt. Die Bestellung erscheint im Adminbereich der Applikation, auf einem anderen Gerät mit mindestens einem der Betriebssysteme
  OS X Yosemite oder Windows 8, und auf dem zusätzlich mindestens einer der Browser Safari 7, Firefox 30,
  Google Chrome oder Internet Explorer 11 installiert ist.

  \textbf{RE-M 08 Adminbereich}\\
  Nachdem eine Speise bestellt worden ist, erscheint ein neuer Bestelleintrag mit Bezeichnung der Speise
  und Tischinformation in einer Liste auf einem PC oder Laptop.

  \textbf{RE-M 09 Responsive Layout}\\
  Das Layout der Speisekarte passt sich automatisch an die Größe des Geräts, von welchem es aufgerufen wird, an.

  \textbf{RE-M 10 Testen der Konzeptstudien}\\
  Um festzustellen ob sich die erarbeiteten Konzepte auch in der Praxis bewähren und somit einen Einsatz zu
  teuren Kameratrackingsystemen schaffen zu können, sind diese umgesetzt. Als Sensoren sind eine Kamera
  sowie Ultraschallsensoren verwendet. Um weitere Positionen zu markieren arbeitet der Multicopter mit
  Farbcodes und/oder Infrarotsendern.

  \textbf{RE-M 11 Konzeptstudie: Sensorik}\\
  Um ohne manuelle Einfüsse fliegen zu können und sein Ziel zu finden braucht der Multicopter eine Reihe
  von Sensoren. Diese dienen zur Positions-bzw. zur Objekterfassung.

  \textbf{RE-M 12 Konzeptstudie: Tischplan}\\
  Um eine vorgegebene Route fliegen zu können muss der Multicopter einen Routenpan vorgegeben bekommen.
  In diesem sind die einzelnen Markierungspunkte enthalten, welche der Multicopter auf seinem Weg zum
  Tisch überfliegt.

  \textbf{RE-M 13 Konzeptstudie: Navigation}\\
  Um zu dem richtigen Tisch zu kommen, muss der Multicopter anhand des Tischplans zu diesem
  navigieren. Im Tischplan sind die Routen und Tischpositionen hinterlegt.

  \textbf{RE-M 14 Konzeptstudie: Kommunikation}\\
  Ein Konzept zur Kommunikation zwischen Multicopter und Sensoren ist entworfen.
  Der Multicopter ist mit den Sensoren verbunden. Von diesen bekommt er die nötigen Informationen.

  \textbf{RE-M 15 Konzeptstudie: Positionserkennung}\\
  Um durch den Raum navigieren zu können muss der Multicopter Informationen zu seiner Position haben.
  Diese kann er anhand einiger Sensoren selbst auswerten und weiterverarbeiten.

  \textbf{RE-M 16 Konzeptstudie: Objekterkennung}\\
  Der Multicopter soll während seinem Flug bestimmte Objekte erkennen können, so z.B. Tische,
  Personen oder Landeplattformen.

  \textbf{RE-M 17 Konzeptstudie: Systemausfall-Maßnahmen}\\
  Tritt bei dem Multicopter ein Systemausfall ein, landet das Flugobjekt und gibt eine Fehlermeldung aus.
  Weiters besteht die Möglichkeit, in den manuellen Flugmodus umzusteigen.

  \textbf{RE-M 18 Konzeptstudie: Sicherheitsmaßnahmen (Software)}\\
  Bei Erkennung eines Hindernisses, welches eine Gefahr für den Multicopter darstellt muss dieser
  entsprechend reagieren. Stationären Objekten wie zum Beispiel Wänden weicht er, sofern dies
  möglich ist, aus. Erkennt er Personen in unmittelbarer Nähe landet er.

  \textbf{RE-M 19 Zusammenbau des Multicopters}\\
  Der ausgewählte Multicopter ist soweit zusammengebaut, dass er flugfähig ist.

  \textbf{RE-M 20 Auswahl der Bauteil (Multicopter)}\\
  Um einen flugfähigen Multicopter zu schaffen, bedarf es mehrerer Bauteile, diese sind ausgewählt
  und bestellt. Dazu zählen unter anderem: Fernbedienung, Gestell, Motoren, Rotorblätter,
  FlightController, Akkus.

  \textbf{RE-M 21 Auswahl der Bauteile (Elektronik)}\\
  Die Sensoren für einen autonomen Flug des Multicopters sind anhand des Sensorkonzepts
  ausgewählt. Weiters ist ein Mikrocontroller ausgewählt, der die Steuerung des Multicopters
  übernimmt.

  \textbf{RE-M 22 Blog Team Nutzerkontos}\\
  Für jedes Teammitglied ist ein Benutzerkonto angelegt, damit individuell Blogeinträge
  verfasst werden können. Zusätzlich zu Namen, kann ein Benutzername, eine e-Mail Adresse
  als Kontaktmöglichkeit und eine "Lieblingsfarbe" angegeben werden.

  \textbf{RE-M 23 Blog Login}\\
  Jedes Teammitglied kann sich mit dem individuell angelegten Benutzerkonto anmelden,
  um Blogeinträge zu verfassen. Nur angemeldeten Benutzern ist es möglich Blogeinträge zu
  verfassen.

  \textbf{RE-M 24 Blog Responsive Design}\\
  Bei Verwendung von kleinen Geräten, wie Smartphones oder Tablets verändert sich die
  Anordnung und Darstellung einzelner Elemente so, dass eine gute Usability auch bei mobilder
  Nutzung des Blogs gewährleistet ist.

  \subsection{Optionale Ziele}
  \textbf{RE-O 01 Speisen verwalten}\\
  Der Adminscreen beinhaltet die Funktion Speisen hinzuzufügen und zu löschen.

  \textbf{RE-O 02 Kategorien}\\
  Bevor der Gast auf den Screen mit den einzelnen Speisen verwiesen worden ist, muss er
  eine Kategorie auswählen, in der die dazugehörigen Speisen gesammelt sind. Anschließend
  sind die Speisen, welche in die ausgewählte Kategorie gehören, aufgelistet.

  \textbf{RE-O 03 Farbcodes für Tische}\\
  Der Admin hat die Möglichkeit den Tischnummern Farbcodes zuzuteilen. Anhand der
  Farbcodes findet der Multicopter die einzelnen Tische.

  \textbf{RE-O 04 Platinenanfertigung}\\
  Die Platinen für die Kommunikation zwischen dem Multicopter, der
  Basis und den Zwischenstationen werden angefertigt und bestückt.

  \textbf{RE-O 05 Sensorik (Hardware)}\\
  Das Positionierungssystem erfordert Sensoren, damit der Multicopter weiß, wo er
  ist, beziehungsweise wo er landen soll. Diese Sensoren sind an dem Multicopter
  mittels Halterungen befestigt.

  \textbf{RE-O 06 Objekttransport - Einfache Transportplatte}\\
  Die Haltevorrichtung ist an dem Multicopter montiert, um ein Objekt transportieren zu können.
  Das Objekt wird auf dem Multicopter platziert und bis zu dem Gast transportiert.
  Die Haltevorrichtung ist möglichst einfach aufgebaut, um Gewicht zu sparen.

  \textbf{RE-O 07 System aufsetzen}\\
  Die einzelnen Komponenten des Multicopters sind miteinander verbunden und er ist
  durch eine manuelle Steuerung flugfähig. Die Entwicklungsumgebung für die
  Programmierung der Software und somit der automatischen Steuerung ist installiert.

  \textbf{RE-O 08 Kommunikation}\\
  Der Multicopter ist mit den ausgewählten Sensoren verbunden. Er kann Befehle geben,
  Informationen empfangen und diese verarbeiten.

  \textbf{RE-O 09 Auswertung Serverdaten}\\
  Der Multicopter empfängt die Tischinformation über Bluetooth/WLAN vom Server,
  wertet diese aus und verarbeitet sie im Anschluss.

  \textbf{RE-O 10 Grundfunktion Multicopter: Starten}\\
  Steigen/Starten, wird durch höhere Drehzahlen erreicht.

  \textbf{RE-O 11 Grundfunktion Multicopter: Landen}\\
  Negative Steigung/Landen, wird durch niedrigere Drehzahlen erreicht.

  \textbf{RE-O 12 Grundfunktion Multicopter: Rollen}\\
  Wenn sich die rechten Propeller schneller als die linken drehen, neigt sichder
  Multicopter nach links und fliegt in diese Richtung. Wenn sich die linken Propeller
  schneller drehen, fliegt er nach rechts.

  \textbf{RE-O 13 Grundfunktion Multicopter: Nicken}\\
  Durch Neigung wird Vortrieb erzeugt. Beim Vorwärtsflug drehen sich die hinteren
  Rotoren schneller, beim Rückwärtsflug die Vorderen.

  \textbf{RE-O 14 Grundfunktion Multicopter: Gieren}\\
  Der Multicopter dreht sich um seine eigene Hochachse, wenn die Drehzahlen der
  Rotorenpaare unterschiedlich sind. Wenn sich die nach links drehenden Rotoren schneller
  bewegen, dreht er sich nach links und umgekehrt nach rechts.

  \textbf{RE-O 15 Blog Entryfilter}\\
  Nutzer des Blogs haben die Möglichkeit sich Blogeinträge nach Autor oder nach
  Art des Eintrags (Blogeintrag oder Entwicklertagebuch) sortieren zu lassen.

  \textbf{RE-O 16 Blog Benutzerkonto bearbeiten}\\
  Es gibt die Möglichkeit für die Teammitglieder Informationen wie Benutzername,
  Farbe oder E-Mail Adresse im Nachhinein zu ändern.

  \textbf{RE-O 17 Blog Statistiken}\\
  Auf dem Dashboard des Blogs sind Informationen, wie zum Beispiel die Anzahl der
  verfassten Blogs, zu sehen. Loggt sich ein Teammitglied ein, sieht es diese Werte
  noch einmal auf die eigenen Blogs bezogen.

  \textbf{RE-O 18 Datenverkehr mit Multicopter}\\
  Durch einen Klick auf den „Servierbutton“, der sich auf der Bestellliste des
  Adminscreens neben jedem Eintrag befindet, bekommt der Multicopter die Tischinformation
  des Gastes, um dessen Bestellung es sich handelt, drahtlos zugeschickt. Dies geschieht
  entweder durch eine Datenübertragung von dem Gerät auf dem der Adminbereich geöffnet
  ist an das Flugobjekt oder durch eine manuelle Eingabe der Tischinformation in eine
  bereits existierende Anwendung, welche die Daten an den Multicopter schickt.

  \subsection{Optionale Erweiterungen}

  \subsection{Nicht-Ziele}

%%%%%%%%%%%%%%%%%%%%%%%%%%%%%%%%%%%%%%%%%%%%%%%%%%%%%%%%%%%%%%%%%%%%%%%%%%%%%%%
\section{Projektmanagement-Methode}

  \subsection{Kanban}

  \subsection{Wasserfall}

  \subsection{Scrum}

%%%%%%%%%%%%%%%%%%%%%%%%%%%%%%%%%%%%%%%%%%%%%%%%%%%%%%%%%%%%%%%%%%%%%%%%%%%%%%%
\section{Teammanagement / Teambuilding}

  \subsection{KaTeCos}

  \subsection{Playground-Meetings}

  \subsection{Sonstiges}
