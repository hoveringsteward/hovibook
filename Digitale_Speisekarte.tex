\chapter{Digitale Speisekarte}
\renewcommand{\kapitelautor}{Autor: Katharina Joksch}

%%%%%%%%%%%%%%%%%%%%%%%%%%%%%%%%%%%%%%%%%%%%%%%%%%%%%%%%%%%%%%%%%%%%%%%%%%%%%%%
\section{Allgemeine technische Planung}

  \subsection{Entwicklungsumgebungen}

    \subsubsection{PhpStorm}
    
PhpStorm ist eine integrierte Entwicklungsumgebung f�r die Programmiersprache PHP, f�r welche man eine kostenpflichtige Lizenz zur Verwendung ben�tigt. Diese Entwicklungsumgebung wurde von JetBrains entwickelt und erschien erstmals 2009 auf dem Markt. Zu den besonderen Features von PhpStorm z�hlen Tools zur Kontrolle der Versionierung, Refaktorisierung, Code- und Syntax-Highlighting. Au�erdem unterst�tzt es PHP-Unit, welches von Symfony verwendet wird und zum Testen von PHP-Skripten dient.

    \subsubsection{Eclipse}
    
Eclipse, entwickelt von der Eclipse Foundation, ist ein Open Source Programmierwerkzeug. Urspr�nglich wurde es als integrierte Entwicklungsumgebung f�r die Programmiersprache Java entwickelt, heutzutage wird es jedoch auch zur Bew�ltigung einiger anderer Entwicklungsaufgaben verwendet.

%%%%%%%%%%%%%%%%%%%%%%%%%%%%%%%%%%%%%%%%%%%%%%%%%%%%%%%%%%%%%%%%%%%%%%%%%%%%%%%
\section{Backend}

  \subsection{Technische Planung}

    \subsubsection{MAMP und XAMPP}

auch zugriff auf mysql erkl�ren

    \subsubsection{Symfony}
    
Symfony ist ein Open Source Web Application Framework, welches das Modell-View-Controller-Schema n�tzt und den Datenbankzugriff mittels einem objektrelationalen Abbild regelt. 

Durch die Einteilung in Modell, View und Controller, ergibt sich beim Entwickeln einer Web Applikation mithilfe von Symfony eine ordentliche Struktur. 
Beim Modell kann man zur Speicherung der Objekte Doctrine verwenden, welches als Bibliothek zur objektrelationalen Abbildung dient.
Die View-Ebene ist f�r die visuelle Darstellung der Applikation zust�ndig. F�r die  Darstellung werden meistens Templates miteinbezogen. Symfony unterst�tzt hierbei die Template Engine Twig. 
Der Controller verwaltet die visuellen Darstellungen der Applikation und nimmt von ihnen Benutzeraktionen entgegen, wertet diese aus und behandelt sie entsprechend. Au�erdem fungiert der Controller als Schnittstelle zwischen Modell und View, was bedeutet, dass er die Daten an die von der einen Schicht zur anderen weiterleitet.

    \subsubsection{Doctrine}

Was ist Doctrine?
bissl Codeschnipsel

    \subsubsection{ER-Modell}

hier kommt dann ein Screenshot von ER-Modell hin + Erkl�rung pks und fks

  \subsection{Umsetzung}
  
    \subsubsection{Framework einrichten}

Terminalbefehle + Erkl�rung

    \subsubsection{Datenbankgenerierung}

Was ich in config.yml und param.yml eingegeben habe
wie das schema meiner db mit code  first durch doctrine ausschaut

    \subsubsection{Datenzugriff}

zugriff mit controller

  \subsection{Herausforderungen und L�sungen}

fk deklarierung
symfony cache l�schen

%%%%%%%%%%%%%%%%%%%%%%%%%%%%%%%%%%%%%%%%%%%%%%%%%%%%%%%%%%%%%%%%%%%%%%%%%%%%%%%
\section{Frontend}

  \subsection{Technische Planung}

    \subsubsection{Bootstrap}

was ist bootstrap? wieso bootstrap? durch klassen layout designen

    \subsubsection{Sass}

was ist sass? sass mit bootstrap? 
durch �ndern der klassenvariablen layout angepasst

    \subsubsection{Gulp}

was ist gulp? wieso gulp und nicht grunt? -> weil gulp schneller

    \subsubsection{Twig}

was macht twig? codeschnipsel?

    \subsubsection{Screen Mockups}

bilder von screenmockups & weshalb schaun sie so aus, aspekte aus designerperspektve aufz�hlen -> usability

  \subsection{Umsetzung}

    \subsubsection{Layout}

Bilder von entg�ltigen screens
bissl was von sass und bootstrap bzw. gulp erkl�ren -> terminal befehle (code nicht, da er eh auf der cd sein wird)

    \subsubsection{Formulargenerierung}

controller & twig

    \subsubsection{Datenausgabe}

twig daten holen von controller 
