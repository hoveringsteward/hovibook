\chapter{Digitale Speisekarte}
\renewcommand{\kapitelautor}{Autor: Katharina Joksch}

%%%%%%%%%%%%%%%%%%%%%%%%%%%%%%%%%%%%%%%%%%%%%%%%%%%%%%%%%%%%%%%%%%%%%%%%%%%%%%%
\section{Allgemeine technische Planung}

  \subsection{Entwicklungsumgebungen}

    \subsubsection{PhpStorm}
PhpStorm ist eine integrierte Entwicklungsumgebung für die Programmiersprache PHP, für welche man eine kostenpflichtige Lizenz zur Verwendung benötigt. Diese Entwicklungsumgebung wurde von JetBrains entwickelt und erschien erstmals 2009 auf dem Markt. Zu den besonderen Features von PhpStorm zählen Tools zur Kontrolle der Versionierung, Refaktorisierung, Code- und Syntax-Highlighting. Außerdem unterstützt es PHP-Unit, welches von Symfony verwendet wird und zum Testen von PHP-Skripten dient.

    \subsubsection{Eclipse}

Eclipse, entwickelt von der Eclipse Foundation, ist ein Open Source Programmierwerkzeug. Ursprünglich wurde es als integrierte Entwicklungsumgebung für die Programmiersprache Java entwickelt, heutzutage wird es jedoch auch zur Bewältigung einiger anderer Entwicklungsaufgaben verwendet.

evtl noch eine extra überschrift für arten von speisekarte also ipad oder iphone + qr code

%%%%%%%%%%%%%%%%%%%%%%%%%%%%%%%%%%%%%%%%%%%%%%%%%%%%%%%%%%%%%%%%%%%%%%%%%%%%%%%
\section{Backend}

  \subsection{Technische Planung}

    \subsubsection{MAMP und XAMPP}

auch zugriff auf mysql erklären

    \subsubsection{Symfony}

Symfony ist ein Open Source Web Application Framework, welches das Model-View-Controller-Schema nützt und den Datenbankzugriff mittels einem objektrelationalen Abbild regelt.

Durch die Einteilung in Model, View und Controller, ergibt sich beim Entwickeln einer Web Applikation mithilfe von Symfony eine ordentliche Struktur.
Beim Model kann man zur Speicherung der Objekte Doctrine verwenden, welches als Bibliothek zur objektrelationalen Abbildung dient.
Die View-Ebene ist für die visuelle Darstellung der Applikation zuständig. Für die  Darstellung werden meistens Templates miteinbezogen. Symfony unterstützt hierbei die Template Engine Twig.
Der Controller verwaltet die visuellen Darstellungen der Applikation und nimmt von ihnen Benutzeraktionen entgegen, wertet diese aus und behandelt sie entsprechend. Außerdem fungiert der Controller als Schnittstelle zwischen Modell und View, was bedeutet, dass er die Daten an die von der einen Schicht zur anderen weiterleitet.


    \subsubsection{Doctrine}

Was ist Doctrine?
bissl Codeschnipsel

    \subsubsection{ER-Modell}

hier kommt dann ein Screenshot von ER-Modell hin + Erklärung pks und fks

  \subsection{Umsetzung}

    \subsubsection{Framework einrichten}

Terminalbefehle + Erklärung

    \subsubsection{Datenbankgenerierung}

Was ich in config.yml und param.yml eingegeben habe
wie das schema meiner db mit code  first durch doctrine ausschaut

    \subsubsection{Datenzugriff}

zugriff mit controller

  \subsection{Herausforderungen und Lösungen}

fk deklarierung
symfony cache löschen

%%%%%%%%%%%%%%%%%%%%%%%%%%%%%%%%%%%%%%%%%%%%%%%%%%%%%%%%%%%%%%%%%%%%%%%%%%%%%%%
\section{Frontend}

  \subsection{Technische Planung}

    \subsubsection{Bootstrap}

was ist bootstrap? wieso bootstrap? durch klassen layout designen

    \subsubsection{Sass}

was ist sass? sass mit bootstrap?
durch ändern der klassenvariablen layout angepasst

    \subsubsection{Gulp}

was ist gulp? wieso gulp und nicht grunt? -> weil gulp schneller

    \subsubsection{Twig}

was macht twig? codeschnipsel?

    \subsubsection{Screen Mockups}

bilder von screenmockups und weshalb schaun sie so aus, aspekte aus designerperspektve aufzählen -> usability

  \subsection{Umsetzung}

    \subsubsection{Layout}

Bilder von entgültigen screens
bissl was von sass und bootstrap bzw. gulp erklären -> terminal befehle (code nicht, da er eh auf der cd sein wird)

    \subsubsection{Formulargenerierung}

controller und twig

    \subsubsection{Datenausgabe}

twig daten holen von controller
