%%%%%%%%%%%%%%%%%%%%%%%%%%%%%%%% BEISPIEL

% im Dokument: alles mit \gls{xxx} kann man anklicken :3

\newglossaryentry{partnership}{
	name=Partnerschaft,
	plural=Partnerschaften,
	description={Um Dateien/Dateiblöcke extern speichern zu können, muss eine
	sogenannte Partnerschaft mit anderen Geräten eingangen werden, bei der man
	Dateiblöcke auf den jeweilig anderen Geräten speichert, während man selber
	Speicherplatz für diese Partner freigibt, in dem deren verschlüsselte
	Dateiblöcke	gespeichert werden}
}

\newglossaryentry{syncpartner}{
	name=Synchronisationspartner,
	plural=Synchronisationspartner,
	description={Gerät mit dem der \sblit-Ordner synchronisiert wird}
}

\newglossaryentry{filecloud}{
	name=Filecloud,
	description={Externer Speicher im Internet, auf dem Dateien gespeichert werden
	können. Meistens bestehend aus einem oder mehreren Servern}
}

%%%%%% Glossar mit Akronym


%\newglossaryentry{gls_aes}{
%	name=Advanced Encryption Standard,
%	description={Standard-Verschlüsselungsverfahren für \glspl{link}},
%	see={[Siehe:]{link}}
%}
%\newacronym[see={[Glossar:]{gls_aes}}]{aes}{AES}{Advanced Encryption Standard\glsadd{gls_aes}}

%\newglossaryentry{gls_gcm}{
%	name=Galois/Counter Mode,
%	description={Standard-Betriebsmodus für die Verschlüsselung von \glspl{link} mit \acrshort{aes}}, %%%% glspl: Link zu Thema "Link" :b (bloedes bsp)
%}
%\newacronym[see={[Glossar:]{gls_gcm}}]{gcm}{GCM}{Galois/Counter Mode\glsadd{gls_gcm}}

%%%%%%%%%%%%%% Unsere Glossareinträge (evtl mit Akronym)

\newglossaryentry{Mitglied}{
	name=Mitglied,
  plural=Mitglieder,
	description={Mitglieder (eng. members) sind Variablen in einer Struktur}
}


%%% im dokument \glslink{Struktur}{Strukturen} muss getestet werden

\newglossaryentry{Struktur}{
	name=Struktur,
  plural=Strukturen,
	description={Eine Struktur (eng. struct) ist eine Speichermethode von Daten. Sie bietet die Möglichkeit einen neuen Datentypen zu erstellen.
  Die einzelnen Variablen/Mitglieder können, anders als bei einem Array, von verschiedenen Datentypen sein. \cite{Structs} }
}

\newglossaryentry{Infrarot}{
	name = Infrarot,
	description = {Infrarot ist ein Teil des Lichtspektrums im nicht sichtbaren Bereich. Die Wellenlänge beträgt dabei zwischen $\SI{1}{\milli\meter}$ und $\SI{800}{\nano\meter}$,
	jene des sichtbaren Lichts liegt zwischen $\SI{800}{\nano\meter}$ und $\SI{400}{\nano\meter}$\cite{Spektrum}, infrarotes Licht ist also langwelliger als sichtbares.}
}
