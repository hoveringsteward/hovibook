% !TEX root = diplomarbeit.tex
\chapter{Marketing}
\renewcommand{\kapitelautor}{Autor: Markus Kaiser}

%%%%%%%%%%%%%%%%%%%%%%%%%%%%%%%%%%%%%%%%%%%%%%%%%%%%%%%%%%%%%%%%%%%%%%%%%%%%%%%
\section{Allgemein}
Im folgenden Kapitel wird beschrieben, mit welchen Mitteln und mit welcher Effektivität
das Projekt Hovering Steward vermarktet worden ist. Unter Anderem werden zwei der wichtigsten Marktanalysen,
aber auch Marketing-Strategien und deren Anwendung erläutert.

  \subsection{Marktanalysen}
  \textbf{SWOT-Analyse}
  Mithilfe der SWOT-Analyse wird der aktuelle Zustand eines Unternehmens, oder in unserem Fall Projektes, analysiert um anschließend eine von vier möglichen Strategien
  für den weiteren Verlauf des Unternehmens zu erarbeiten. Es werden Stärken, Schwächen, Möglichkeiten und Risiken evaluiert, wobei sich jeweils Stärken \& Schwächen, und
  Möglichkeiten \& Risiken gegenüberstehen. Die vier Begriffe auf englisch übersetzt Strengths, Weaknesses, Opportunities und Risks ergeben dadurch das Akronym SWOT.

  \textcolor{red}{(BILD Analyse)}

  Die vier Strategien:
  \begin{itemize}
    \item \textbf{SO-Strategie}
    Die Stärken des Unternehmens werden hervorgehoben und mit den Möglichkeiten verknüpft. Macht ein Unternehmen beispielsweise hohe Umsätze in Industrieländern und hat
    genügend finanzielle Mittel, wird expandiert und ein neuer Standort in einem weiteren Industrieland aufgebaut.

    \item \textbf{ST-Strategie}
    Auch hier werden die Stärken des Unternehmens genutzt, allerdings um bestehende Risiken einzuschränken oder zu entfernen. Besteht die Gefahr, dass die Konkurrenz
    einen höheren Marktanteil erlangt, kann das Unternehmen beispielsweise durch eine sehr gute Marketingabteilung neue Produkte herstellen, die dies verhindern.

    \item \textbf{WO-Strategie}
    Hierbei wird versucht, die Schwächen eines Unternehmens als Chance zu sehen sich weiterzuentwickeln. Erzielt ein gewisses Produkt nicht die erwarteten Gewinne,
    können durch neue Lösungsansätze neue Erfahrungen gewonnen werden, die nach Erfolg auch auf andere Produkte angewendet werden können.

    \item \textbf{WT-Strategie}
    Die Schwächen des Unternehmens werden näher betrachtet, um beim Versuch diese zu dezimieren, gleichzeitig Riskiken einzugrenzen. Diese Strategie kann sehr hilfreich
    sein, wenn das Unternehmen sich auf dem Markt schwer über Wasser halten kann.

  \end{itemize}
  \textbf{Portfolioanalyse nach Boston Consulting Group}
  Die Definition laut {"Boston Consulting Group"\cite{portfolioanalyse}} beschreibt:
  "Das BCG-Portfolio erlaubt eine Bewertung strategisch relevanter Geschäftseinheiten auf Basis zukünftiger Gewinnchancen (Marktwachstum) und der
  gegenwärtigen Wettbewerbsposition (relativer Marktanteil)." Anders gesagt: Diese Analyse hilft dabei Teile eines Unternehmens wie Produkte oder Dienstleistungen
  nach Kosten und Ertrag einzuschätzen und in vier Kategorien zu unterteilen.

  (BILD BCG-Matrix)

  \begin{itemize}
    \item Die \textbf{Cashcow} ist ein Produkt des Unternehmens, welches sich bereits lange Zeit auf dem Markt hält und dadurch hohe Gewinne erziehlt. Durch die
    erarbeitete hohe Position auf dem Markt decken sich ihre Kosten durch den Gewinn selbst. Die Cashcow stellt zudem die Basis für die Weiterentwicklung anderer
    Produkte dar.

    \item Der \textbf{Superstar} ist ein Produkt mit den besten Aussichten. Es erzielt bereits nach kurzer Zeit auf dem Markt hohe Gewinne. Um den Marktanteil
    zu halten, muss jedoch viel in dieses Produkt investiert werden. In der Regel geschieht dies durch die Überschüsse, die die Cashcow abwirft.

    \item Ein \textbf{Poor Dog} ist ein weniger erfolgreiches Produkt. Da es nach hohen Investitionen und langer Zeit auf dem Markt keine nennenswerte
    Gewinne erbracht hat, ist es ratsam zu überlegen, ob weiter investiert, oder die Produktion gestoppt wird.

    \item Als \textbf{Questionmarks} werden die Produkte bezeichnet, bei denen sich noch nicht eindeutig gezeigt hat, in welche Richtung sie sich entwickeln werden.
    Es steht offen ob sie zu einem Superstar, oder einem Poor Dog werden. Diese Tatsache macht es schwer zu entscheiden, ob weiter in sie investiert, oder das Produkt
    verkauft wird.

  \end{itemize}
  \subsection{Marketing-Strategie}



%%%%%%%%%%%%%%%%%%%%%%%%%%%%%%%%%%%%%%%%%%%%%%%%%%%%%%%%%%%%%%%%%%%%%%%%%%%%%%%
\section{Blog}
Dieses Kapitel setzt sich mit dem Entwicklungsprozess, von technischer Planung bis Testphase
des Blogs auseinander. Es werden technische Hintergründe wie verwendete Technologien, aber
auch allgemeine Überlegungen wie der Einsatz eines bereits vorhandenen Content Management Systems
behandelt.

  \subsection{Technische Planung}
  ...
  Die technische Planung setzt sich aus folgenden Teilbereichen zusammen:
    \begin{itemize}
      \item Eigenes und vorhandenes CMS
      \item Server und Datenbank
      \item Mockups
      \item Frameworks
    \end{itemize}

    \subsubsection{Eigenes und vorhandenes CMS}
    Der Blog fungiert als Content Management System, kurz CMS. Der Zweck eines
    Content Management Systems ist der, dass durch Trennung von Inhalt und Design eine
    sehr einfache Nutzung ermöglicht wird. Gewöhnliche Nutzer müssen keine besonderen
    Kenntnisse über das Bearbeiten oder Erstellen einer Website vorweisen, sondern
    können über einfache Eingabefelder Text, Bilder oder Videos auf dem Blog einbinden.
    ...

    \subsubsection{Server und Datenbank}


    Für die Speicherung der Nutzerdaten des Teams und . (Aufzählung von Alternativen und Begründung)

    \subsubsection{Mockups}
    Die Überlegungen bezüglich des Designs des Blogs unterschieden sich stark zwischen den ersten
    Entwürfen und der schlussendlich gewählten Variante.

    Die Wahl der Farben basierte ursprünglich auf einer Studie der deutschen Schriftstellerin Eva Heller,
    die mit ihrem Buch "Wie Farben wirken" analysierte, welche Farbtöne von einem Großteil der Menschen
    mit welcher Eigenschaft verbunden werden. Der Fokus von Hovering Steward lag darin, die zwei Begriffe {"Hunger"\cite{WieFarbenWirken}}
    und "Appetit" möglichst gut zu vermitteln. Das Resultat war eine Kombination der Farben Braun, Gelb und Rosa
    assoziiert wurden. Den Blog mithilfe dieser Farbkombination zu gestalten erschien jedoch nicht passend, weswegen
    ein zweiter Ansatz notwendig war.

    Die Überlegung den Blog gleich wie die digitale Speisekarte zu gestalten führte dazu, Elemente, die an ein Restaurant
    erinnern einzubinden. Dies führte zu der Idee, einen Holztisch im Hintergrund abzubilden.
    Damit das Design trotzdem einen modernen Eindruck macht wurde für die Darstellung von Inhalten ein {"Flat Design"\cite{FlatDesign}}
    herangezogen. Das Flat Design zeichnet sich vor allem durch sehr vereinfache Gestaltung, grelle Farben und zweidimensionale Abbildungen aus.
    Es ist auf optimale Usability, also verständliche Nutzung ausgelegt.

    Das Endresultat war folgendes:
    (BILD VOM BLOG)
    \subsubsection{Frameworks}
    Für die Entwicklung des Blogs erschien der Einsatz eines Frameworks als sehr hilfreich. Das Framework sollte komplexe Methoden wie
    Datenbankzugriffe oder Formularvalidierungen vereinfacht zur Verfügung stellen. Eine Recherche brachte folgende Frameworks als
    Optionen:
    \begin{itemize}
      \item Laravel
      \item Symphony
      \item CodeIgniter
      \item CakePHP
    \end{itemize}

    Die Wahl fiel letztlich auf CodeIgniter
  \subsection{Umsetzung}

    \subsubsection{Frontend}
    Das Frontend des Blogs setzt sich aus den drei Technologien HTML, Javascript und CSS zusammen.

    \subsubsection{Backend}

    \subsubsection{SEO}
    Der Blog war unsere wichtigste Schnittstelle nach Außen, weswegen er im Internet leicht zu finden sein musste. Suchmaschinen wie Google oder Bing nutzen
    sogenannte Robots, kleine Programme die selbstständig das World Wide Web durchstöbern, um Websiten zu inspizieren und zu bewerten. Je besser die Bewertung,
    desto weiter oben lässt sich die Website in der Suchmaschine auffinden.
    Die Bewertung hängt von einer Vielzahl von Kriterien ab, von denen folgende bei der entwicklung des Blogs beachtet wurden:

    \begin{itemize}
      \item \textbf{Aufbau der Seite}\\
        Die Grundstruktur eines HTML Dokuments besteht aus folgenden Tags:
        (BILD VON HTML BASIC STRUCTURE)
        Im <head> Bereich werden zum einen <meta> Tags angegeben, in denen allgemeine Informationen zur Website, wie Autor oder Keywords und Beschreibung gespeichert werden.
        Letztere zwei sind besonders wichtig, da sie dem Robot mitteilen, um was es auf der Seite geht. Der Robot achtet hierbei auch darauf, dass die Beschreibung nicht
        länger als 155 Zeichen lang ist und die Keywords eindeutig gewählt werden.

        Im <body>, also im eigentlichen Inhalt der Website legt der Robot zum einen sehr viel Wert Struktur. Beispielsweise muss die Verschachtelung der Tags stimmen, oder der <h1> Tag darf nur einmal pro Seite
        vorkommen. Zum anderen werden Inhalt und Code in Relation gesetzt. Wenn der Aufbau des HTML Dokuments riesig ist, in jedem Absatz jedoch nur ein Satz steht kennt sich der Robot nicht aus.

      \item \textbf{Responsive Design}\\
        Das Smartphone ist mittlerweile ein ebenso beliebtes Gerät zum Surfen im Internet, wie der Computer. Damit auf dem kleinen Display alles gut lesbar muss die Website speziell
        gestaltet werden. Frameworks wie Bootsrap helfen bei diesem Vorhaben, indem mithilfe von vodefinierten CSS-Klassen die Elemente der Website so gestaltet werden, dass diese mit der
        Bilschirmgröße mitwachsen, sich verschieben oder ausgeblendet werden.

      \item \textbf{Interne und externe Links}\\
        Bei Links werden sowohl Links, die auf die Website verweisen, als auch Links, die auf andere Seiten zeigen bewertet.
        Zeigen sehr viele externe Links auf eine Seite, interpretiert der Robot diese als geläufig und sie erhält eine höhere Wertung. In unserem Fall war die Kombination mit anderen sozialen
        Medien sehr hilfreich, da durch Veröffentlichungen auf diesen automatisch Links auf unseren Blog generiert werden.
        Links, die von der eigenen Seite auf andere verweisen stellen jedoch einen Nachteil dar, da der Robot dem Link folgen könnte, statt die eigene Seite weiter zu inspizieren. Möchte man dies
        verhindern, kann dem <a> Tag das Attribut rel="nofollow" hinzugefügt werden, um dem Robot zu sagen, dass er diesem Link nicht folgen soll.

      \item \textbf{Sitemap}\\
      	Die Sitemap ist ein Inhaltsverzeichnis der Website, z.B. in Form einer XML Datei. Die Sitemap hilft dem Robot, wenn er erst einmal auf einer Seite gelandet ist, zugehörige Seiten
        zu finden. Dadurch, dass in der Sitemap die zugehörigen Unterseiten einer Seite mittels URL angegeben werden, hat dies den selben Vorteil wie eine große Anzahl externer Links, die auf diese
        Unterseiten verweisen, nämlich dass Google diese schneller findet. Tools wie xml-sitemaps.com bieten an, eine Sitemap für die angegebene URL zu generieren. Hat der Robot die Seite vollständig
        untersucht, muss die Sitemap im Domain Root-Verzeichnis abgelegt werden. Zuletzt kann die URL zu dieser Sitemap bei Google registriert werden, um Google anzuweisen einen Robot über die Seite
        laufen zu lassen. Nach einigen Tagen sind die Ergebnisse direkt in der Google Suche sichtbar.

      \item \textbf{Browserkompatibilität}\\
        Nicht jeder Browser stellt jede Seite gleich dar. Die Tags werden unterschiedlich, oder sogar garnicht verstanden. Wird dies nicht beachtet, wirkt sich das auf die Nutzerfreundlichkeit,
        und damit auch auf die Bewertung der Website aus. Um dies zu verhindern gibt es Tools wie zum Beispiel caniuse.com in denen genau verzeichnet ist, welcher Tag, oder welche CSS-Eigenschaft
        von welchem Browser, beziehungsweise welcher Browserversion unterstützt wird.

      \item \textbf{Ladezeit}\\
        Der Robot bewertet ebenfalls die Ladezeit einer Website. Sie ergibt sich aus der Anzahl an Bildern, Videos oder eingebundenen Bibliotheken oder Schriftarten auf der Website.
        Speziell im Blog, war früh klar, dass viele Bilder geladen werden müssen. Den Upload auf eine niedrige Dateigröße einzuschränken hat zu Beginn nicht ausgereicht.
        Eine Javascript Bibliothek names yoxview hat es allerdings ermöglicht, die Bilder erst dann zu laden, wenn ein Nutzer auf einen Thumbnail klickt. Die Größe der Thumbnails betrugen bloß
        ein paar Kilobyte, wodurch die Ladezeit der Website bedeutend eingeschränkt werden konnte.

      \item \textbf{Barrierefreiheit}\\
        Es gibt verschiedene Möglichkeiten eine Website so zu optimieren, dass auch körperlich eingeschränkte Nutzer sie selbstständig verwenden können.


    \end{itemize}
  \subsection{Implementierung}

    \subsubsection{Code-Beispiele}

    \subsubsection{Testing}

  \subsection{Herausforderungen und Lösungen}

    \subsubsection{Sicherheit}

    \subsubsection{Responsive Design}

%%%%%%%%%%%%%%%%%%%%%%%%%%%%%%%%%%%%%%%%%%%%%%%%%%%%%%%%%%%%%%%%%%%%%%%%%%%%%%%
\section{Social Media}

  \subsection{Technische Planung}

    \subsubsection{Corporate Design}

    \subsubsection{Corporate Identity}

  \subsection{Umsetzung}

    \subsubsection{Blogposts}

    \subsubsection{Facebook}

    \subsubsection{Twitter}

  \subsection{Herausforderungen und Lösungen}

    \subsubsection{Konsistenz zwischen den Netzwerken}

%%%%%%%%%%%%%%%%%%%%%%%%%%%%%%%%%%%%%%%%%%%%%%%%%%%%%%%%%%%%%%%%%%%%%%%%%%%%%%%
\section{Wettbewerbe, Events, Präsentationen}

  \subsection{Technische Planung}

    \subsubsection{Präsentationsauftritt}

    \subsubsection{Marketing-Artikel}

  \subsection{Umsetzung}

    \subsubsection{Jugend Innovativ}

    \subsubsection{Jahr der Forschung}

    \subsubsection{Bosch - Technik fürs Leben Preis}

    \subsubsection{Accenture}

    \subsubsection{Invasion der Drohnen}

    \subsubsection{Tag der offenen Türe}

    \subsubsection{European Conference on Educational Robotics}
