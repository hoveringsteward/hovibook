\chapter{Marketing}
\renewcommand{\kapitelautor}{Autor: Markus Kaiser}

%%%%%%%%%%%%%%%%%%%%%%%%%%%%%%%%%%%%%%%%%%%%%%%%%%%%%%%%%%%%%%%%%%%%%%%%%%%%%%%
\section{Allgemein}
Im folgenden Kapitel wird beschrieben, mit welchen Mitteln und mit welcher Effektivität
das Projekt Hovering Steward vermarktet wird. Unter Anderem werden diverse Marktanalysen
aber auch Marketing-Strategien und deren Anwendung erläutert.
  \subsection{Marktanalyse}

  \subsection{Marketing-Strategie}

%%%%%%%%%%%%%%%%%%%%%%%%%%%%%%%%%%%%%%%%%%%%%%%%%%%%%%%%%%%%%%%%%%%%%%%%%%%%%%%
\section{Blog}
Dieses Kapitel setzt sich mit dem Entwicklungsprozess, von technischer Planung bis Testphase
des Blogs auseinander. Es werden technische Hintergründe wie verwendete Technologien, aber
auch allgemeine Überlegungen wie der Einsatz eines bereits vorhandenen Content Management Systems
behandelt.
  \subsection{Technische Planung}
  Ganz allgemein war die Idee des Blogs
  Die technische Planung setzt sich aus folgenden Teilbereichen zusammen:
    \begin{itemize}
      \item Eigenes und vorhandenes CMS
      \item Server und Datenbank
      \item Mockups
      \item Frameworks
    \end{itemize}
    \subsubsection{Eigenes und vorhandenes CMS}
    Der Blog fungiert als Content Management System, kurz CMS. Der Zweck eines
    Content Management Systems ist der, dass duch Trennung von Inhalt und Design eine
    sehr einfache Nutzung ermöglicht wird. Gewöhnliche Nutzer müssen keine besonderen
    Kenntnisse über das Bearbeiten oder Erstellen einer Website vorweisen, sondern
    können über einfache Eingabefelder Text, Bilder oder Videos auf dem Blog einbinden.
    Chrisy ist wunderschön

    \subsubsection{Server und Datenbank}
    Der Blog verwendet eine MySQL Datenbank. (Aufzählung von Alternativen und Begründung)
    \subsubsection{Mockups}
    Die Überlegungen bezüglich des Designs des Blogs unterschieden sich stark zwischen den ersten
    Entwürfen und der schlussendlich gewählten Variante.

    Die Wahl der Farben basierte ursprünglich auf einer Studie der deutschen Schriftstellerin Eva Heller,
    die mit ihrem Buch \'Wie Farben wirken\' analysierte, welche Farbtöne von einem Großteil der Menschen
    mit welcher Eigenschaft verbunden werden. Der Fokus von Hovering Steward lag darin, die zwei Begriffen {\'Hunger\'\cite{WieFarbenWirken}}
    und \'Appetit\' möglichst gut zu vermitteln. Das Resultat war eine Kombination der Farben Braun, Gelb und Rosa
    assoziiert wurden. Den Blog mithilfe dieser Farbkombination zu gestalten erschien jedoch nicht passend, weswegen
    ein zweiter Ansatz notwendig war.

    Die Überlegung den Blog gleich wie die digitale Speisekarte zu gestalten führte dazu, Elemente, die an ein Restaurant
    erinnern einzubinden. Dies führte zu der Idee, einen Holztisch im Hintergrund abzubilden.
    Damit das Design trotzdem einen modernen Eindruck macht wurde für die Darstellung von Inhalten ein {\'Flat Design\'\cite{FlatDesgin}}
    herangezogen. Das Flat Design zeichnet sich vorallem durch sehr vereinfache Gestaltung, grelle Farben und zweidimensionale Abbildungen aus.
    Es ist auf optimale Usability, also verständliche Nutzung ausgelegt.

    Das Endresultat war folgendes:
    (BILD VOM BLOG)
    \subsubsection{Frameworks}
    Für die Entwicklung des Blogs erschien der Einsatz eines Frameworks als sehr hilfreich. Das Framework sollte komplexe Methoden wie
    Datenbankzugriffe oder Formularvalidierungen vereinfacht zur Verfügung stellen. Eine Recherche brachte folgende Frameworks als
    Optionen:
    \begin{itemize}
      \item Laravel
      \item Symphony
      \item CodeIgniter
      \item CakePHP
    \end{itemize}

    Die Wahl fiel letztlich auf CodeIgniter
  \subsection{Umsetzung}

    \subsubsection{Frontend}

    \subsubsection{Backend}

    \subsubsection{SEO}

  \subsection{Implementierung}

    \subsubsection{Code-Beispiele}

    \subsubsection{Testing}

  \subsection{Herausforderungen und Lösungen}

    \subsubsection{Sicherheit}

    \subsubsection{Responsive Design}

%%%%%%%%%%%%%%%%%%%%%%%%%%%%%%%%%%%%%%%%%%%%%%%%%%%%%%%%%%%%%%%%%%%%%%%%%%%%%%%
\section{Social Media}

  \subsection{Technische Planung}

    \subsubsection{Corporate Design}

    \subsubsection{Corporate Identity}

  \subsection{Umsetzung}

    \subsubsection{Blogposts}

    \subsubsection{Facebook}

    \subsubsection{Twitter}

  \subsection{Herausforderungen und Lösungen}

    \subsubsection{Konsistenz zwischen den Netzwerken}

%%%%%%%%%%%%%%%%%%%%%%%%%%%%%%%%%%%%%%%%%%%%%%%%%%%%%%%%%%%%%%%%%%%%%%%%%%%%%%%
\section{Wettbewerbe, Events, Präsentationen}

  \subsection{Technische Planung}

    \subsubsection{Präsentationsauftritt}

    \subsubsection{Marketing-Artikel}

  \subsection{Umsetzung}

    \subsubsection{Jugend Innovativ}

    \subsubsection{Jahr der Forschung}

    \subsubsection{Bosch - Technik fürs Leben Preis}

    \subsubsection{Accenture}

    \subsubsection{Invasion der Drohnen}

    \subsubsection{Tag der offenen Türe}

    \subsubsection{European Conference on Educational Robotics}
